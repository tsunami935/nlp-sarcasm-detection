\documentclass[11pt]{article}

\usepackage{authblk}
\usepackage{hyperref}
\usepackage[utf8]{inputenc}
\usepackage{amsmath}
\usepackage{amsfonts}
\usepackage{amssymb}
\usepackage{siunitx}
\usepackage{graphicx}
\usepackage{subcaption}
\usepackage{float}
\usepackage[nottoc,numbib]{tocbibind}
\usepackage{biblatex}

\bibliography{ref.bib}

\newcommand{\email}[1]{\texttt{\href{mailto:#1}{#1}}}

\title{COMP4420 Project Proposal: Sarcasm Detection}
\author{
    Bui, Nam \\
    \email{nam\_bui@student.uml.edu}
    % \and
    % Put names here
}

\begin{document}

\maketitle

\section{Introduction}

% Sentiment analysis

% The difficulty of determining sarcasm
% e.g. sentence in one context may be sarcastic while in another context may be serious

\section{Dataset}

The dataset used will be a collection of
tagged newspaper headlines \cite{misra2023Sarcasm}.

% Structure of data in dataset
% When and how dataset was collected

\section{Evaluation Method}

% Since it's classification, F1 is a decent metric

\printbibliography

\end{document}